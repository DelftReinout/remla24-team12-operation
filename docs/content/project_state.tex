% Insufficient -> The extension is unrelated to the release engineering practices and focuses on an implementation aspect. The extension is trivial or irrelevant for the current project.
% Sufficient -> The report describes a meaningful extension to either the training pipeline, release pipeline, contribution process, or deployment. It does not have to cover more than one.
% Good -> The report contains a critical reflection on the existing design and provides a convincing argumentation of a current shortcoming and its negative effect.
% Very Good -> The presented solution is clearly addressing the described shortcoming. The report also includes a brief explanation of how the success could be measured objectively.
% Excellent -> The presented extension is general in nature. It is relevant and applicable beyond the context of the concrete project.

\section{Project State}
As is only natural, no project is only smooth sailing. Every design decision has its risks and pitfalls. In this section we therefore reflect on the current state of the project, discuss what went wrong and how to improve.
\subsection{Limitations}
% Critically reflect on the current state of your project and identify the point that you find the most critical/annoying/error prone. Equipped with the knowledge of the course.
% Describe the identified shortcoming and its effect. A convincing argumentation is crucial.


\subsection{Extension} % or refactoring
% • Describe and visualize a project refactoring/extension that improves the situation.
% • Link to information sources that provide useful information about the problem, inspiration for your solution, or concrete examples for its realization. We expect that you cite respectable sources (e.g., research papers; quality blogs like Medium; tool websites; or popular StackOverflow discussions).
% • Describe how you could test whether the changed design would solve the original shortcoming.
TODO